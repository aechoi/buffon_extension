\documentclass{article}



\usepackage{arxiv}

\usepackage[utf8]{inputenc} % allow utf-8 input
\usepackage[T1]{fontenc}    % use 8-bit T1 fonts
\usepackage{hyperref}       % hyperlinks
\usepackage{url}            % simple URL typesetting
\usepackage{booktabs}       % professional-quality tables
\usepackage{amsfonts}       % blackboard math symbols
\usepackage{nicefrac}       % compact symbols for 1/2, etc.
\usepackage{microtype}      % microtypography
\usepackage{lipsum}		% Can be removed after putting your text content
\usepackage{graphicx}
\usepackage{natbib}
\usepackage{doi}
\usepackage{amsmath}
\usepackage{amsthm}
\usepackage{tikz}
\newtheorem{theorem}{Theorem}
\newtheorem{proposition}{Proposition}



\title{Multidimensional Extension of Buffon's Needle Problem}

%\date{September 9, 1985}	% Here you can change the date presented in the paper title
%\date{} 					% Or removing it

\author{ Alexander~Choi\\
	alexander.e.choi@gmail.com
}

% Uncomment to remove the date
%\date{}

% Uncomment to override  the `A preprint' in the header
%\renewcommand{\headeright}{Technical Report}
%\renewcommand{\undertitle}{Technical Report}
\renewcommand{\shorttitle}{Multidimensional Buffon's Needle}

%%% Add PDF metadata to help others organize their library
%%% Once the PDF is generated, you can check the metadata with
%%% $ pdfinfo template.pdf
\hypersetup{
pdftitle={Multidimensional Extension of Buffon's Needle Problem},
pdfsubject={q-bio.NC, q-bio.QM},
pdfauthor={Alexander E.~Choi},
pdfkeywords={Buffon's needle problem, Geometric probability},
}

\begin{document}
\maketitle

\begin{abstract}
	Consider a line segment randomly placed on a two-dimensional plane ruled with a set of regularly spaced parallel lines. The classical Buffon's needle problem
    asks what the probability is that the line segment intersects at least 1 of these lines. This paper extends this problem by considering a line segment randomly
    placed in $\mathbb{R}^D$ and its probability of intersection with a set of regularly spaced parallel hyperplanes. 
\end{abstract}


% keywords can be removed
\keywords{Buffon's needle problem \and Geometric Probability}

\section{Introduction}
Buffon's needle problem was originally posed in the 18th century with the following premise. Given a line segment, or "needle", of length $r$ randomly dropped on a two-dimensional plane
ruled with a set of parallel lines regularly spaced $s$ units apart, what is the probability that the needle crosses at least 1 of the lines? The solution, it turns out, is
$\frac{2r}{s\pi}$ when $r<s$. Variations and extensions of this problem have been investigated as well, including
\begin{itemize}
	\item Laplace's Extension - Investigating when the plane is gridded with 2 orthogonal sets of parallel lines with spacings $s_1$ and $s_2$.
	\item Buffon's Noodle - Instead of being rigidly straight, the needle is permitted to bend (a "noodle").
	\item Pivot Needle - The needle is constructed of two line segments that hinge together. Each crossing is considered.
\end{itemize}

In this paper, we investigate a particular extension that allows the needle to be dropped into a space with dimension greater than 2. In these higher dimensions, we will rule the space
with parallel hyperplanes rather than lines. Additionally, we will look at gridding the space with orthogonal sets of hyperplanes, thereby extending Laplace's extension into higher dimensions.

Given $D\in\mathbb{N}_{>0}$ and $N\in[1,2,\dots,D]$, consider a grid on $\mathbb{R}^D$ formed by $N$ orthogonal sets of regularly spaced hyperplanes where each set of hyperplanes
has a potentially unique spacing of $S_i$. For example, if $D=2, N=1, S_1=2$, the grid would match the original Buffon Needle problem and would have only a single set of parallel lines 2 units apart.
If $D=2, N=2, S=[1, 2]$, the grid would have 2 sets of parallel lines that are orthogonal to each other, matching the problem in Laplace's extension. One set of lines would have a spacing of 1 unit and 
the other would have a spacing of 2 units.
% Insert diagram of two examples here
\begin{tikzpicture}
	
\end{tikzpicture}

A line segment of length $r\in\mathbb{R}^+$ is randomly located in the space such that one of its end points, $P_0$, is uniformly distributed
across the entire domain. The line segment's orientation is independently distributed such that when considering $P_0$ as the center of a $(D-1)$-sphere of radius $r$, the other point, $P_1$,
is uniformly distributed on the surface of that hypersphere. This line segment may intersect with $C\in\mathbb{N}$ unique hyperplanes. This paper studies the probability that the line segment
intersects with at least $c$ hyperplanes, $P(C\ge c|r, D, N, S)$. From there, solutions for crossing less than $c$ hyperplanes and exactly $c$ hyperplanes can
be derived.

We will define the coordinates of line segment using $\vec{x}\in\mathbb{R}^D$ for the location of $P_0$ and spherical coordinates for the location of $P_1$ with respect to $P_0$.

\begin{align*}
    y_1 &= r\cos{\phi_1}\\
    y_2 &= r\sin{\phi_1}\cos{\phi_2}\\
    \vdots\\
    y_{D-1} &= r\sin{\phi_1}\hdots\sin{\phi_{D-2}}\cos{\phi_{D-1}}\\
    y_{D} &= r\sin{\phi_1}\hdots\sin{\phi_{D-2}}\sin{\phi_{D-1}}\\
    P_1 &= \vec{x} + \vec{y}\\
	\phi_j &\in \begin{cases}[0, \pi] & j<D-1 \\ [0, 2\pi] & j=D-1\end{cases}
\end{align*}

Translational symmetry of the grid of hyperplanes allows us to consider the domain of $P_0$ to be $x_i\in[0,S_i]$ as the origin can be moved to any point on the grid.
Reflectional symmetry of the grid also allows us to consider the domain of $\vec{y}$ to be a single orthant of the hypersphere. For convenience, we will pick the orthant where
$\phi_i \in [0, \pi/2]$.

The rest of the paper is organized as follows. A derivation of the joint probability density function for $P_0$ and $P_1$ will be provided in \S 2. The derivation and validation
of the crossing probabilities for $N=1$ will be given in \S 3. The derivation and validation of the crossing probabiities for any $N$ and $r<\min(S)$ will be given in \S 4. Analysis of the limits
and extrema of the probabilities is explored in \S 4.

\section{Joint Probability Density of the Line Segment}
Each coordinate for $P_0$ can be defined as a uniformly distributed random variable $X_i\sim{\rm Uniform}(0,S_i)$. Due to independence, the joint PDF for $P_0$ is the product
$\prod_{i=1}^D \frac{1}{S_i}$. By the definition of the problem, the coordinates $\vec{x}$ do not influence the orientation of the line segment defined by $\vec{\phi}$. 
The probability density function for the uniform distribution of points on an orthant of the hypersphere can be determined by calculating the area element in terms of spherical coordinates.

\begin{proposition}
In spherical coordinates, the probability density function for a uniform distribution on an orthant of a hypersphere is $\frac{2^D}{A_{D-1}}\prod_{j=1}^{D-1}r\sin^{D-1-j}\phi_j$ where
$A_{D-1}$ is the surface area of a $(D-1)$-sphere.
\end{proposition}
\begin{proof}
	The area element of an $(D-1)$-sphere of radius $r$ can be expressed as 
	\begin{equation} \label{eq:1}
	d\Omega = \left(\prod_{j=1}^{D-1}r\sin^{D-1-j}\phi_j\right)d\phi_1 \hdots d\phi_{D-1}
	\end{equation}
	The probability that a point lies in this differential element can be expressed as follows.
	\begin{equation} \label{eq:2}
	f_{\Omega}(\Omega)d\Omega = f_{\vec{\phi}}(\phi_1, \hdots, \phi_{D-1})d\phi_1 \hdots d\phi_{D-1}
	\end{equation}
	The points are uniformly distributed over the surface of an orthant of the hypersphere implying that $f_{\Omega}(\Omega) = \frac{2^D}{A_{D-1}}$. Substituting this and
	\ref{eq:1} into \ref{eq:2} yields
	\begin{equation}
	\frac{2^D}{A_{D-1}}\prod_{j=1}^{D-1}r\sin^{D-1-j}\phi_j = f_{\vec{\phi}}(\phi_1, \hdots, \phi_{D-1})
	\end{equation}
\end{proof}

Then by independence, the joint probability density function for the entire line segment can be expressed as 
\begin{equation} \label{eq:general pdf}
	f_{\vec{X},\vec{\phi}}(x_1, \hdots, x_D, \phi_1, \hdots, \phi_{D-1}) = \frac{2^D}{A_{D-1}}\left(\prod_{i=1}^D\frac{1}{S_i}\right)\left(\prod_{j=1}^{D-1}r\sin^{D-1-j}\phi_j\right)
\end{equation} 

\section{Probability of crossing $N=1$}
In general, the probability of meeting some number of crossings given any set of parameters can be described as follows.

\begin{align} 
	P(C\ge c|r, D, N, S)&=\idotsint_V f_{\vec\phi}(\phi_1,\hdots,\phi_{D-1})dx_1 \hdots dx_D d\phi_1 \hdots d\phi_{D-1}\\
	&= \frac{2^Dr^{D-1}}{A_{D-1}\prod_{i=1}^D S_i} \idotsint_V \prod_{j=1}^{D-1}\sin^{D-1-j}\phi_j dx_1\hdots dx_D d\phi_1 \hdots d\phi_{D-1} \label{eq:volume integral}
\end{align}

Where $V$ is the hypervolume in which some sort of crossing condition is met. The definition of these crossing conditions and the solution to the above equation will be
explored for a variety of parameters.

We start with a simplified set of parameters where there is only a single set of parallel hyperplanes. We are interested in the condition where at least $c$ crossings happen. 
That is, in this section we are interested in the probability $P(C\ge c | r, D, N=1, S)$. For brevity, we will refer to this as $P_1(c)$.

Due to rotational symmetry of the line segment, it should not matter in which direction the hyperplanes extend. Without loss of generality we assume the planes are in the
direction of $x_1$.

Because $P_0$ is constrained to be within the gridcell at the origin and because the orthant we are investigating is in the direction of $x_1$, we know that a crossing occurs
whenever the following condition is met
\begin{align}
	x_1 &+ r\cos{\phi_1} > S_1c\\
	r &> \frac{S_1c - x_1}{\cos{\phi_1}} \\
	x_1 &> S_1c - r\cos{\phi_1} \\
	\phi_1 &< \arccos{\frac{S_1c-x_1}{r}}
\end{align}

The minimum of $\frac{S_1c - x_1}{\cos{\phi_1}}$ occurs at $x_1=S_1, \phi_1=0$ with a value of $S_1(c-1)$. Therefore if $r<S_1(c-1)$ we can guarantee that the crossing condition
cannot be satisfied. This results in
\begin{equation}
	P(C\ge c|r<S_1(c-1), N=1) = 0
\end{equation}

The domains of $x_1$ can then be used to define the space in which a valid crossing has occured 
\begin{align} 
	m(\phi_1) &< x_1 < S_1 \\ \label{eq:crossing condition 0}
	m(\phi_1) &= \max(0, S_1c-r\cos{\phi_1}) 
\end{align}

The domain of $x_1$ also provides a maximum bound for the maximum acceptable value of $\phi_1$ when $x_1 = S_1$.
\begin{equation}
	\phi_1 < \arccos{\frac{S_1(c-1)}{r}}
\end{equation}


We can now express our volume integral in terms of these conditions and solve for the locatcion dimensions.
\begin{align} \label{eq:volume integral}
	P_1(c) &= \int_0^{\pi/2} \hdots \int_0^{\pi/2} \int_0^{\arccos{\frac{S_1(c-1)}{r}}}\int_0^{S_D} \hdots \int_0^{S_2} \int_{m(\phi_1)}^{S_1} f_{\vec\phi}(\phi_1,\hdots,\phi_{D-1})dx_1 dx_2 \hdots dx_D d\phi_1 d\phi_2 \hdots d\phi_{D-1}\\
	&= \int_0^{\pi/2} \hdots \int_0^{\pi/2}\int_0^{\arccos{\frac{S_1(c-1)}{r}}}\left(\prod_{i=2}^DS_i\right)\left(S_1c-m(\phi_1)\right)  f_{\vec\phi}(\phi_1,\hdots,\phi_{D-1}) d\phi_1 d\phi_2 \hdots d\phi_{D-1} \\
	&= \frac{ 2^Dr^{D-1} \prod_{i=2}^DS_i}{A_{D-1}\prod_{i=1}^DS_i}\int_0^{\pi/2} \hdots \int_0^{\pi/2}\int_0^{\arccos{\frac{S_1(c-1)}{r}}} (S_1c-m(\phi_1)) \prod_{j=1}^{D-1}\sin^{D-1-j}\phi_j d\phi_1 d\phi_2\hdots d\phi_{D-1}\\
	&= \frac{ 2^Dr^{D-1}}{A_{D-1}S_1}\int_0^{\pi/2} \hdots \int_0^{\pi/2}\int_0^{\arccos{\frac{S_1(c-1)}{r}}} (S_1c-m(\phi_1)) \prod_{j=1}^{D-1}\sin^{D-1-j}\phi_j d\phi_1 d\phi_2 \hdots d\phi_{D-1}
\end{align}

The value of $m(\phi_1)$ depends on the value of $r$. If $r<S_1c$, then $m(\phi_1)=S_1c-r\cos \phi_1 \forall \phi_1$. If $r>S_1c$ we will need to partition the interval of integration into
two regions, one where $S_1c-r\cos{\phi_1}$ is greater than 0 and one where it is less than zero. The transition occurs at the value $\phi_1 = \arccos{\frac{S_1c}{r}}$.
\begin{equation}
	m(\phi_1) = \begin{cases}
		0 & \frac{S_1c}{\cos\phi_1}>r>S_1c\\
		S_1c - r\cos{\phi_1} & \text{otherwise}
	\end{cases}
\end{equation}

\subsection{$S_1(c-1)<r<S_1c$}
When $r<S_1c$ we have the following expression by substituting $S_1c - r\cos{\phi_1}$ for $m(\phi_1)$.
\begin{align}
	P_1(c) &= \frac{ 2^Dr^{D-1} }{A_{D-1}S_1}\int_0^{\pi/2} \hdots \int_0^{\pi/2}\int_0^{\arccos{\frac{S_1(c-1)}{r}}} (S_1(1-c)+r\cos\phi_1) \prod_{j=1}^{D-1}\sin^{D-1-j}\phi_j d\phi_1 d\phi_2\hdots d\phi_{D-1}\\
	&= \frac{2^D r^{D-1}}{A_{D-1} S_1}\int_0^{\pi/2} \hdots \int_0^{\pi/2}\int_0^{\arccos{\frac{S_1(c-1)}{r}}} (S_1(1-c)+ r\cos\phi_1)\sin^{D-2}\phi_1 \prod_{j=2}^{D-1}\sin^{D-1-j}\phi_j d\phi_1 d\phi_2 \hdots d\phi_{D-1} \\
	\begin{split} \label{eq:first two part}
		&= \frac{2^D r^{D}}{A_{D-1} S_1}\int_0^{\pi/2} \hdots \int_0^{\pi/2}\prod_{j=2}^{D-1}\sin^{D-1-j}\phi_j\bigg(\frac{S_1(1-c)}{r}\int_0^{\arccos{\frac{S_1(c-1)}{r}}} \sin^{D-2}\phi_1  d\phi_1 \\
		&\qquad + \int_0^{\arccos{\frac{S_1(c-1)}{r}}} \cos\phi_1\sin^{D-2}\phi_1  d\phi_1 \bigg)d\phi_2 \hdots d\phi_{D-1}
	\end{split}
\end{align}

The two interior integrals can be solved via integration by reduction and u-substitution respectively. It is convenient if we first define the following proposition.
\begin{proposition} \label{prop:double fac beta}
	When given the ratio $(k-1)!!/k!!$ where the double exclam represents the double factorial function, it is equivalent the following.
	\begin{equation}
		= \begin{cases}
			\frac{1}{\pi}B(\frac{k+1}{2}, \frac{1}{2}) & k\bmod 2=0\\
			\frac{1}{2}B(\frac{k+1}{2}, \frac{1}{2}) & k\bmod 2=1
		\end{cases}
	\end{equation}
\end{proposition}
\begin{proof}
	We start by deriving a value for $n!!$ in terms of factorials. If $n\bmod 2 = 0$
	\begin{align}
		n!! &= n(n-2)\hdots(4)(2) \\
		&= 2^{n/2}\frac{n}{2}\frac{n-2}{2}\hdots\frac{4}{2}\frac{2}{2} \\
		&= 2^{n/2}\frac{n}{2}! \label{eq:double factorial even}
	\end{align}
	If $n\bmod 2= 1$
	\begin{align}
		n!! &= n(n-2)\hdots(3)(1) \\
		&= \frac{n!}{(n-1)!!} \\
		&= \frac{n!}{2^{(n-1)/2}(\frac{n-1}{2})!} \label{eq:double factorial odd}
	\end{align}

	Using \ref{eq:double factorial even} and \ref{eq:double factorial odd} we can simplify $(k-1)!!/k!!$. First, assuming that $k$ is even
	\begin{align}
		\frac{(k-1)!!}{k!!} &= \frac{(k-1)!}{2^{(k-2)/2}(\frac{k-2}{2})!}\frac{1}{2^{k/2}(\frac{k}{2})!} \\
		&= \frac{\Gamma(\frac{1}{2})}{\Gamma(\frac{1}{2})}\frac{\frac{1}{2}\frac{2}{2}\hdots\frac{k-2}{2}\frac{k-1}{2}}{\frac{2}{2}\frac{4}{2}\hdots\frac{k-4}{2}\frac{k-2}{2}(\frac{k}{2}!)} \\
		&= \frac{\Gamma(\frac{1}{2})}{\Gamma(\frac{1}{2})}\frac{\frac{1}{2}\frac{3}{2}\hdots\frac{k-3}{2}\frac{k-1}{2}}{\frac{k}{2}!}
	\end{align}
	Now using the property $n\Gamma(n)=\Gamma(n+1)$ and $n! = \Gamma(n+1)$, we get the following.
	\begin{equation}
		\frac{(k-1)!!}{k!!} = \frac{\Gamma(\frac{k+1}{2})}{\Gamma(\frac{1}{2})\Gamma(\frac{k+2}{2})}
	\end{equation}
	Finally, using $B(x,y)=\Gamma(x)\Gamma(y)/\Gamma(x+y)$ and $\Gamma(1/2) = \sqrt{\pi}$ we get
	\begin{equation}
		\frac{(k-1)!!}{k!!} = \frac{1}{\pi}B\left(\frac{k+1}{2}, \frac{1}{2}\right)
	\end{equation}

	We now repeat the process for the case where $k$ is odd.
	\begin{align}
		\frac{(k-1)!!}{k!!} &= \left(\frac{k-1}{2}\right)!2^{(k-1)/2}\frac{(\frac{k-1}{2})!2^{(k-1)/2}}{k!} \\
		&= \frac{2\Gamma(\frac{1}{2})}{2\Gamma(\frac{1}{2})}\frac{2^{k-1}(\frac{k-1}{2})!^2}{k!} \\
		&= \frac{\Gamma(\frac{1}{2})}{2\Gamma(\frac{1}{2})}\frac{\frac{2}{2}\frac{4}{2}\hdots\frac{k-3}{2}\frac{k-1}{2}(\frac{k-1}{2}!)}{\frac{1}{2}\frac{2}{2}\hdots\frac{k-1}{2}\frac{k}{2}} \\
		&= \frac{\Gamma(\frac{1}{2})}{2\Gamma(\frac{1}{2})}\frac{\frac{k-1}{2}!}{\frac{1}{2}\frac{3}{2}\hdots\frac{k-2}{2}\frac{k}{2}} \\
		&= \frac{\Gamma(\frac{1}{2})\Gamma(\frac{k+1}{2})}{2\Gamma(\frac{k+2}{2})} \\
		&= \frac{1}{2}B\left(\frac{k+1}{2}, \frac{1}{2}\right)
	\end{align}
\end{proof}

We now define the following proposition for the initial integral in \ref{eq:first two part}.
\begin{proposition} \label{prop:sin integral}
	Any integral of the form $\int_0^{\arccos(\gamma)} \sin^m \phi d\phi$ has two possible solutions depending on the parity of $m$.
	\begin{align}
		& = \frac{B(\frac{m+1}{2}, \frac{1}{2})}{2}\left(g(\gamma, m) - \gamma(1-\gamma^2)^{(m+1)/2} \sum_{i=1}^{\lfloor m/2 \rfloor}\frac{B(\frac{m+2-2i}{2}, \frac{1}{2})}{\pi(1-\gamma^2)^i}\right) \\
		& g(\gamma, m)=\begin{cases}
			\frac{2}{\pi}\arccos \gamma & m \bmod 2=0 \\
			1 - \gamma & m \bmod 2=1
		\end{cases}
	\end{align}
\end{proposition}
\begin{proof}
	We start with the following integration by reduction identity
	\begin{align}
		\int_0^{\arccos \gamma}\sin^m\phi d\phi &= - \frac{1}{m}\sin^{m-1}\phi \cos \phi \Big{|}_0^{\arccos\gamma} + \frac{m-1}{m}\int_0^{\arccos \gamma}\sin^{m-2}\phi d\phi \\
		\begin{split}
			&= -\frac{1}{m}(1-\gamma^2)^{(m-1)/2}\gamma \\
			&\qquad + \frac{m-1}{m}\left(-\frac{1}{m-2}\sin^{m-3}\phi\cos\phi\Big{|}_0^{\arccos\gamma} + \frac{m-3}{m-2}\int_0^{\arccos\gamma}\sin^{m-4}\phi d\phi \right)
		\end{split}
	\end{align}
	This pattern continues until the $\sin$ in the final integrand is raised to either the first or zeroth power. This depends on whether $m$ is even or odd. If $m$ is even
	\begin{align}
		\begin{split}
			&= -\frac{1}{m}(1-\gamma^2)^{(m-1)/2}\gamma - \frac{m-1}{m(m-2)}(1-\gamma^2)^{(m-3)/2}\gamma - \hdots \\
			&\qquad \qquad - \frac{(m-1)(m-3)\hdots(3)}{(m)(m-2)\hdots(2)}(1-\gamma^2)^{1/2}\gamma+ \frac{(m-1)!!}{m!!}\int_0^{\arccos\gamma} d\phi
		\end{split} \\
		&= \frac{(m-1)!!}{m!!}\left(-\frac{(m-2)!!}{(m-1)!!}(1-\gamma^2)^{(m-1)/2}\gamma - \frac{(m-4)!!}{(m-3)!!}(1-\gamma^2)^{(m-3)/2}\gamma - \hdots - \frac{0!!}{1!!}(1-\gamma^2)^{1/2}+ \arccos\gamma\right)\\
		&= \frac{(m-1)!!}{m!!}\left(\arccos\gamma-\gamma\sum_{i=1}^{m/2}\frac{(m-2i)!!}{(m+1-2i)!!}(1-\gamma^2)^{(m+1-2i)/2} \right)\\
		&= \frac{(m-1)!!}{m!!}\left(\arccos\gamma-\gamma(1-\gamma^2)^{(m+1)/2}\sum_{i=1}^{m/2}\frac{(m-2i)!!}{(m+1-2i)!!}(1-\gamma^2)^{-i} \right)
	\end{align}
	Using \ref{prop:double fac beta} we can reduce to the following.
	\begin{align}
		&= \frac{B(\frac{m+1}{2}, \frac{1}{2})}{\pi}\left(\arccos\gamma-\gamma(1-\gamma^2)^{(m+1)/2}\sum_{i=1}^{m/2}\frac{B(\frac{m+2-2i}{2}, \frac{1}{2})}{2}(1-\gamma^2)^{-i} \right) \\
		&= \frac{B(\frac{m+1}{2}, \frac{1}{2})}{2}\left(\frac{2}{\pi}\arccos\gamma-\gamma(1-\gamma^2)^{(m+1)/2}\sum_{i=1}^{m/2}\frac{B(\frac{m+2-2i}{2}, \frac{1}{2})}{\pi(1-\gamma^2)^{i}} \right)
	\end{align}

	Repeating for the case where $m$ is odd
	\begin{align}
		\begin{split}
			&= -\frac{1}{m}(1-\gamma^2)^{(m-1)/2}\gamma - \frac{m-1}{m(m-2)}(1-\gamma^2)^{(m-3)/2}\gamma - \hdots \\
			&\qquad \qquad - \frac{(m-1)(m-3)\hdots(3)}{(m)(m-2)\hdots(2)}(1-\gamma^2)^{1/2}\gamma+ \frac{(m-1)!!}{m!!}\int_0^{\arccos\gamma} \sin\phi d\phi
		\end{split} \\
		&= \frac{(m-1)!!}{m!!}\left(1-\gamma-\gamma(1-\gamma^2)^{(m+1)/2}\sum_{i=1}^{(m-1)/2}\frac{(m-2i)!!}{(m+1-2i)!!}(1-\gamma^2)^{-i} \right) \\
		&= \frac{B(\frac{m+1}{2}, \frac{1}{2})}{2}\left(1-\gamma-\gamma(1-\gamma^2)^{(m+1)/2}\sum_{i=1}^{\lfloor m/2 \rfloor}\frac{B(\frac{m+2-2i}{2}, \frac{1}{2})}{\pi(1-\gamma^2)^{i}} \right)
	\end{align}
\end{proof}

We can substitute the solution from \ref{prop:sin integral} int \ref{eq:first two part} to get the following.
\begin{equation}
	\begin{split}
		P_1(c) &= \frac{2^D r^{D}}{A_{D-1} S_1}\int_0^{\pi/2} \hdots \int_0^{\pi/2}\prod_{j=2}^{D-1}\sin^{D-1-j}\phi_j\bigg(\frac{\gamma}{2}B\left(\frac{D-1}{2}, \frac{1}{2}\right)\left(g(\gamma, D-2)-\gamma(1-\gamma^2)^{(D-1)/2}\sum_{k=1}^{\lfloor \frac{D-2}{2} \rfloor}\frac{B(\frac{D-2k}{2}, \frac{1}{2})}{\pi(1-\gamma^2)^k}\right) \\
		&\qquad + \int_0^{\arccos{\gamma}} \cos\phi_1\sin^{D-2}\phi_1  d\phi_1 \bigg)d\phi_2 \hdots d\phi_{D-1}
	\end{split}
\end{equation}
Where $\gamma = S_1(c-1)/r$.


Applying u-substitution where $u=\sin\phi_1$ we get the following
\begin{align} 
	P_1(c) &= \frac{2^D r^D \xi}{A_{D-1} S_1}\int_0^{\pi/2} \hdots \int_0^{\pi/2} \prod_{j=2}^{D-1}\sin^{D-1-j}\phi_i d\phi_2\hdots d\phi_{D-1} \\ \label{eq:final phi integral}
	\xi &= \frac{\gamma}{2}B\left(\frac{D-1}{2}, \frac{1}{2}\right)\left(g(\gamma, D-2)-\gamma(1-\gamma^2)^{(D-1)/2}\sum_{k=1}^{\lfloor \frac{D-2}{2} \rfloor}\frac{B(\frac{D-2k}{2}, \frac{1}{2})}{\pi(1-\gamma^2)^k}\right) + \frac{1}{D-1}(1-\gamma^2)^{(D-1)/2} \\
	&= \frac{1}{D-1}\left(\frac{\pi\gamma}{B(\frac{D}{2}, \frac{1}{2})}\left(g(\gamma, D-2)-\gamma(1-\gamma^2)^{(D-1)/2}\sum_{k=1}^{\lfloor \frac{D-2}{2} \rfloor}\frac{B(\frac{D-2k}{2}, \frac{1}{2})}{\pi(1-\gamma^2)^k}\right) + (1-\gamma^2)^{(D-1)/2}\right) \label{eq:xi}
\end{align}

To solve the remaining $D-2$ integrals, we start by noting that we can simplify the result from \ref{prop:sin integral}
by noting that the remaining upper bounds of integration are all $\pi/2$. 

Restating the result, we have the following
\begin{align}
	\int_0^{\arccos \gamma}sin^m\phi d\phi &= \frac{1}{2}B\left(\frac{m+1}{2}, \frac{1}{2}\right)\left(g(\gamma, m) - \gamma\sum_{i=1}^{\lfloor m/2 \rfloor}\frac{B(\frac{m+2-2i}{2}, \frac{1}{2})}{\pi}(1-\gamma^2)^{(m+1)/2-i}\right) \\
	\int_0^{\arccos 0}sin^m\phi d\phi &= \frac{1}{2}B\left(\frac{m+1}{2}, \frac{1}{2}\right)(1-0) \\
	&= \frac{1}{2}B\left(\frac{m+1}{2}, \frac{1}{2}\right)
\end{align}

For every integral in \ref{eq:final phi integral} we get the following product.
\begin{align}
	P_1(c) &= \frac{2^D r^D \xi}{A_{D-1} S_1}\prod_{j=2}^{D-1}\frac{1}{2}B\left(\frac{j+1}{2}, \frac{1}{2}\right) \\
	&= \frac{2^D r^D \xi}{A_{D-1} S_1} \frac{\Gamma(\frac{3}{2})}{\Gamma(\frac{D+1}{2})}\left(\frac{\sqrt{\pi}}{2}\right)^{D-2}
\end{align}

We can now substitute in an expression of $A_{D-1}$ as follows
\begin{align}
	A_{D-1} &= \frac{2\pi^{D/2}r^{n-1}}{\Gamma(\frac{D}{2})}\\
	P_1(c) &= \frac{2^D r^D \xi}{2\pi^{D/2}r^{n-1}S_1} \frac{\Gamma(\frac{D}{2})\Gamma(\frac{3}{2})}{\Gamma(\frac{D+1}{2})}\left(\frac{\sqrt{\pi}}{2}\right)^{D-2} \\
	&= \frac{2 r \xi}{\pi S_1} \frac{\Gamma(\frac{D}{2})\Gamma(\frac{1}{2})\frac{1}{2}}{\Gamma(\frac{D+1}{2})} \\
	&= \frac{r \xi}{\pi S_1}B\left(\frac{D}{2}, \frac{1}{2}\right)
\end{align}
Substituting in our function for $\xi$ in \ref{eq:xi}
\begin{align}
	P_1(c) &= \frac{r}{S_1}\left(\frac{\gamma}{D-1}\right)
\end{align}
% Need to verify and make correct


\subsubsection{$r>S_1$}
When $r>S_1$ we must split the bounds of integration for the conditions where $\phi_1<\arccos{\frac{S_1}{r}}$ and $\phi_1>\arccos{\frac{S_1}{r}}$. 
\begin{align}
	P(C>0|r, D, N=1, S) &= \frac{r^D 2^D}{S_1 A_{D-1}}\int_0^{\pi/2} \hdots \int_0^{\arccos{S_1}{r}} \cos\phi_1 \prod_{j=1}^{D-1}\sin^{D-1-j} \phi_i d\phi_1\hdots d\phi_{D-1}\\
	&= \frac{r^D 2^D}{S_1 A_{D-1}}\int_0^{\pi/2} \hdots \int_0^{\pi/2} \cos\phi_1\sin^{D-2}\phi_1 \prod_{j=2}^{D-1}\sin^{D-1-j}\phi_i d\phi_1\hdots d\phi_{D-1}
\end{align}

%Below here is template stuff
\section{Headings: first level}
\label{sec:headings}

\lipsum[4] See Section \ref{sec:headings}.

\subsection{Headings: second level}
\lipsum[5]
\begin{equation} 
	\xi _{ij}(t)=P(x_{t}=i,x_{t+1}=j|y,v,w;\theta)= {\frac {\alpha _{i}(t)a^{w_t}_{ij}\beta _{j}(t+1)b^{v_{t+1}}_{j}(y_{t+1})}{\sum _{i=1}^{N} \sum _{j=1}^{N} \alpha _{i}(t)a^{w_t}_{ij}\beta _{j}(t+1)b^{v_{t+1}}_{j}(y_{t+1})}}
\end{equation}



\subsubsection{Headings: third level}
\lipsum[6]

\paragraph{Paragraph}
\lipsum[7]



\section{Examples of citations, figures, tables, references}
\label{sec:others}

\subsection{Citations}
Citations use \verb+natbib+. The documentation may be found at
\begin{center}
	\url{http://mirrors.ctan.org/macros/latex/contrib/natbib/natnotes.pdf}
\end{center}

Here is an example usage of the two main commands (\verb+citet+ and \verb+citep+): Some people thought a thing \citep{kour2014real, hadash2018estimate} but other people thought something else \citep{kour2014fast}. Many people have speculated that if we knew exactly why \citet{kour2014fast} thought this\dots

\subsection{Figures}
\lipsum[10]
See Figure \ref{fig:fig1}. Here is how you add footnotes. \footnote{Sample of the first footnote.}
\lipsum[11]

\begin{figure}
	\centering
	\fbox{\rule[-.5cm]{4cm}{4cm} \rule[-.5cm]{4cm}{0cm}}
	\caption{Sample figure caption.}
	\label{fig:fig1}
\end{figure}

\subsection{Tables}
See awesome Table~\ref{tab:table}.

The documentation for \verb+booktabs+ (`Publication quality tables in LaTeX') is available from:
\begin{center}
	\url{https://www.ctan.org/pkg/booktabs}
\end{center}


\begin{table}
	\caption{Sample table title}
	\centering
	\begin{tabular}{lll}
		\toprule
		\multicolumn{2}{c}{Part}                   \\
		\cmidrule(r){1-2}
		Name     & Description     & Size ($\mu$m) \\
		\midrule
		Dendrite & Input terminal  & $\sim$100     \\
		Axon     & Output terminal & $\sim$10      \\
		Soma     & Cell body       & up to $10^6$  \\
		\bottomrule
	\end{tabular}
	\label{tab:table}
\end{table}

\subsection{Lists}
\begin{itemize}
	\item Lorem ipsum dolor sit amet
	\item consectetur adipiscing elit.
	\item Aliquam dignissim blandit est, in dictum tortor gravida eget. In ac rutrum magna.
\end{itemize}


\bibliographystyle{unsrtnat}
\bibliography{references}  %%% Uncomment this line and comment out the ``thebibliography'' section below to use the external .bib file (using bibtex) .


%%% Uncomment this section and comment out the \bibliography{references} line above to use inline references.
% \begin{thebibliography}{1}

% 	\bibitem{kour2014real}
% 	George Kour and Raid Saabne.
% 	\newblock Real-time segmentation of on-line handwritten arabic script.
% 	\newblock In {\em Frontiers in Handwriting Recognition (ICFHR), 2014 14th
% 			International Conference on}, pages 417--422. IEEE, 2014.

% 	\bibitem{kour2014fast}
% 	George Kour and Raid Saabne.
% 	\newblock Fast classification of handwritten on-line arabic characters.
% 	\newblock In {\em Soft Computing and Pattern Recognition (SoCPaR), 2014 6th
% 			International Conference of}, pages 312--318. IEEE, 2014.

% 	\bibitem{hadash2018estimate}
% 	Guy Hadash, Einat Kermany, Boaz Carmeli, Ofer Lavi, George Kour, and Alon
% 	Jacovi.
% 	\newblock Estimate and replace: A novel approach to integrating deep neural
% 	networks with existing applications.
% 	\newblock {\em arXiv preprint arXiv:1804.09028}, 2018.

% \end{thebibliography}


\end{document}

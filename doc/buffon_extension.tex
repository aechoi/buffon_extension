\documentclass{article}



\usepackage{arxiv}

\usepackage[utf8]{inputenc} % allow utf-8 input
\usepackage[T1]{fontenc}    % use 8-bit T1 fonts
\usepackage{hyperref}       % hyperlinks
\usepackage{url}            % simple URL typesetting
\usepackage{booktabs}       % professional-quality tables
\usepackage{amsfonts}       % blackboard math symbols
\usepackage{nicefrac}       % compact symbols for 1/2, etc.
\usepackage{microtype}      % microtypography
\usepackage{lipsum}		% Can be removed after putting your text content
\usepackage{graphicx}
\usepackage{natbib}
\usepackage{doi}
\usepackage{amsmath}



\title{Multidimensional Extension of Buffon's Needle Problem}

%\date{September 9, 1985}	% Here you can change the date presented in the paper title
%\date{} 					% Or removing it

\author{ Alexander~Choi\\
	alexander.e.choi@gmail.com
}

% Uncomment to remove the date
%\date{}

% Uncomment to override  the `A preprint' in the header
%\renewcommand{\headeright}{Technical Report}
%\renewcommand{\undertitle}{Technical Report}
\renewcommand{\shorttitle}{Multidimensional Buffon's Needle}

%%% Add PDF metadata to help others organize their library
%%% Once the PDF is generated, you can check the metadata with
%%% $ pdfinfo template.pdf
\hypersetup{
pdftitle={Multidimensional Extension of Buffon's Needle Problem},
pdfsubject={q-bio.NC, q-bio.QM},
pdfauthor={Alexander E.~Choi},
pdfkeywords={Buffon's needle problem, Geometric probability},
}

\begin{document}
\maketitle

\begin{abstract}
	Consider a line segment randomly placed on a two-dimensional plane ruled with a set of regularly spaced parallel lines. The classical Buffon's needle problem
    asks what the probability is that the line segment intersects at least 1 of these lines. This paper extends this problem by considering a line segment randomly
    placed in $\mathbb{R}^D$ and its probability of intersection with a set of regularly spaced parallel hyperplanes. 
\end{abstract}


% keywords can be removed
\keywords{Buffon's needle problem \and Geometric Probability}

\section{Introduction}
Given $D\in\Bbb{N}_{>0}$ and $N\in[1,2,\dots,D]$, consider a grid on $\Bbb{R}^D$ formed by $N$ orthogonal sets of regularly spaced hyperplanes. Each set of hyperplanes
has a potentially unique spacing of $S_i$. A line segment of length $l\in\Bbb{R}^+$ is randomly located in the space such that one of its end points, $P_0$, is uniformly distributed
across the entire domain. The line segments orientation is distributed such that when considering $P_0$ as the center of a $(D-1)$-sphere of radius $l$, the other point, $P_1$,
is uniformly distributed on the surface of that hypersphere. This line segment may intersect with $C\in\Bbb{N}$ unique hyperplanes. This paper studies the probability of
the line segment intersecting more than $c$ hyperplanes, $P(C>c|l, D, N, S)$. From there, solutions for crossing less than $c$ hyperplanes and exactly $c$ hyperplanes can
be derived.

As an example, the classical Buffon's needle problem would be $P(C>0|l, 2, 1, S)$. Laplace's extension would be represented as $P(C>0|l, 2, 2, S)$.

The orientation of the line segment can be represented using spherical coordinates of a $(D-1)$-sphere. 

\begin{align}
    x_1 &= r\cos{\phi_1}\\
    x_2 &= r\sin{\phi_1}\cos{\phi_2}\\
    \vdots\\
    x_{n-1} &= r\sin{\phi_1}\hdots\sin{\phi_{D-2}}\cos{\phi_{D-1}}\\
    x_{n} &= r\sin{\phi_1}\hdots\sin{\phi_{D-2}}\sin{\phi_{D-1}}\\
    \vec{y} &= \vec{x} + P_0
\end{align}

There are several symmetries which simplify the problem. Translational symmetry of the grid of hyperplanes allows us to consider the domain of $P_0$ to be $P_0\in[0,S_i]^D$
as the origin can be moved to any point on the grid.

\section{}


\section{Headings: first level}
\label{sec:headings}

\lipsum[4] See Section \ref{sec:headings}.

\subsection{Headings: second level}
\lipsum[5]
\begin{equation}
	\xi _{ij}(t)=P(x_{t}=i,x_{t+1}=j|y,v,w;\theta)= {\frac {\alpha _{i}(t)a^{w_t}_{ij}\beta _{j}(t+1)b^{v_{t+1}}_{j}(y_{t+1})}{\sum _{i=1}^{N} \sum _{j=1}^{N} \alpha _{i}(t)a^{w_t}_{ij}\beta _{j}(t+1)b^{v_{t+1}}_{j}(y_{t+1})}}
\end{equation}

\subsubsection{Headings: third level}
\lipsum[6]

\paragraph{Paragraph}
\lipsum[7]



\section{Examples of citations, figures, tables, references}
\label{sec:others}

\subsection{Citations}
Citations use \verb+natbib+. The documentation may be found at
\begin{center}
	\url{http://mirrors.ctan.org/macros/latex/contrib/natbib/natnotes.pdf}
\end{center}

Here is an example usage of the two main commands (\verb+citet+ and \verb+citep+): Some people thought a thing \citep{kour2014real, hadash2018estimate} but other people thought something else \citep{kour2014fast}. Many people have speculated that if we knew exactly why \citet{kour2014fast} thought this\dots

\subsection{Figures}
\lipsum[10]
See Figure \ref{fig:fig1}. Here is how you add footnotes. \footnote{Sample of the first footnote.}
\lipsum[11]

\begin{figure}
	\centering
	\fbox{\rule[-.5cm]{4cm}{4cm} \rule[-.5cm]{4cm}{0cm}}
	\caption{Sample figure caption.}
	\label{fig:fig1}
\end{figure}

\subsection{Tables}
See awesome Table~\ref{tab:table}.

The documentation for \verb+booktabs+ (`Publication quality tables in LaTeX') is available from:
\begin{center}
	\url{https://www.ctan.org/pkg/booktabs}
\end{center}


\begin{table}
	\caption{Sample table title}
	\centering
	\begin{tabular}{lll}
		\toprule
		\multicolumn{2}{c}{Part}                   \\
		\cmidrule(r){1-2}
		Name     & Description     & Size ($\mu$m) \\
		\midrule
		Dendrite & Input terminal  & $\sim$100     \\
		Axon     & Output terminal & $\sim$10      \\
		Soma     & Cell body       & up to $10^6$  \\
		\bottomrule
	\end{tabular}
	\label{tab:table}
\end{table}

\subsection{Lists}
\begin{itemize}
	\item Lorem ipsum dolor sit amet
	\item consectetur adipiscing elit.
	\item Aliquam dignissim blandit est, in dictum tortor gravida eget. In ac rutrum magna.
\end{itemize}


\bibliographystyle{unsrtnat}
\bibliography{references}  %%% Uncomment this line and comment out the ``thebibliography'' section below to use the external .bib file (using bibtex) .


%%% Uncomment this section and comment out the \bibliography{references} line above to use inline references.
% \begin{thebibliography}{1}

% 	\bibitem{kour2014real}
% 	George Kour and Raid Saabne.
% 	\newblock Real-time segmentation of on-line handwritten arabic script.
% 	\newblock In {\em Frontiers in Handwriting Recognition (ICFHR), 2014 14th
% 			International Conference on}, pages 417--422. IEEE, 2014.

% 	\bibitem{kour2014fast}
% 	George Kour and Raid Saabne.
% 	\newblock Fast classification of handwritten on-line arabic characters.
% 	\newblock In {\em Soft Computing and Pattern Recognition (SoCPaR), 2014 6th
% 			International Conference of}, pages 312--318. IEEE, 2014.

% 	\bibitem{hadash2018estimate}
% 	Guy Hadash, Einat Kermany, Boaz Carmeli, Ofer Lavi, George Kour, and Alon
% 	Jacovi.
% 	\newblock Estimate and replace: A novel approach to integrating deep neural
% 	networks with existing applications.
% 	\newblock {\em arXiv preprint arXiv:1804.09028}, 2018.

% \end{thebibliography}


\end{document}
